% Options for packages loaded elsewhere
\PassOptionsToPackage{unicode}{hyperref}
\PassOptionsToPackage{hyphens}{url}
%
\documentclass[
]{article}
\usepackage{amsmath,amssymb}
\usepackage{iftex}
\ifPDFTeX
  \usepackage[T1]{fontenc}
  \usepackage[utf8]{inputenc}
  \usepackage{textcomp} % provide euro and other symbols
\else % if luatex or xetex
  \usepackage{unicode-math} % this also loads fontspec
  \defaultfontfeatures{Scale=MatchLowercase}
  \defaultfontfeatures[\rmfamily]{Ligatures=TeX,Scale=1}
\fi
\usepackage{lmodern}
\ifPDFTeX\else
  % xetex/luatex font selection
\fi
% Use upquote if available, for straight quotes in verbatim environments
\IfFileExists{upquote.sty}{\usepackage{upquote}}{}
\IfFileExists{microtype.sty}{% use microtype if available
  \usepackage[]{microtype}
  \UseMicrotypeSet[protrusion]{basicmath} % disable protrusion for tt fonts
}{}
\makeatletter
\@ifundefined{KOMAClassName}{% if non-KOMA class
  \IfFileExists{parskip.sty}{%
    \usepackage{parskip}
  }{% else
    \setlength{\parindent}{0pt}
    \setlength{\parskip}{6pt plus 2pt minus 1pt}}
}{% if KOMA class
  \KOMAoptions{parskip=half}}
\makeatother
\usepackage{xcolor}
\usepackage[margin=1in]{geometry}
\usepackage{color}
\usepackage{fancyvrb}
\newcommand{\VerbBar}{|}
\newcommand{\VERB}{\Verb[commandchars=\\\{\}]}
\DefineVerbatimEnvironment{Highlighting}{Verbatim}{commandchars=\\\{\}}
% Add ',fontsize=\small' for more characters per line
\usepackage{framed}
\definecolor{shadecolor}{RGB}{248,248,248}
\newenvironment{Shaded}{\begin{snugshade}}{\end{snugshade}}
\newcommand{\AlertTok}[1]{\textcolor[rgb]{0.94,0.16,0.16}{#1}}
\newcommand{\AnnotationTok}[1]{\textcolor[rgb]{0.56,0.35,0.01}{\textbf{\textit{#1}}}}
\newcommand{\AttributeTok}[1]{\textcolor[rgb]{0.13,0.29,0.53}{#1}}
\newcommand{\BaseNTok}[1]{\textcolor[rgb]{0.00,0.00,0.81}{#1}}
\newcommand{\BuiltInTok}[1]{#1}
\newcommand{\CharTok}[1]{\textcolor[rgb]{0.31,0.60,0.02}{#1}}
\newcommand{\CommentTok}[1]{\textcolor[rgb]{0.56,0.35,0.01}{\textit{#1}}}
\newcommand{\CommentVarTok}[1]{\textcolor[rgb]{0.56,0.35,0.01}{\textbf{\textit{#1}}}}
\newcommand{\ConstantTok}[1]{\textcolor[rgb]{0.56,0.35,0.01}{#1}}
\newcommand{\ControlFlowTok}[1]{\textcolor[rgb]{0.13,0.29,0.53}{\textbf{#1}}}
\newcommand{\DataTypeTok}[1]{\textcolor[rgb]{0.13,0.29,0.53}{#1}}
\newcommand{\DecValTok}[1]{\textcolor[rgb]{0.00,0.00,0.81}{#1}}
\newcommand{\DocumentationTok}[1]{\textcolor[rgb]{0.56,0.35,0.01}{\textbf{\textit{#1}}}}
\newcommand{\ErrorTok}[1]{\textcolor[rgb]{0.64,0.00,0.00}{\textbf{#1}}}
\newcommand{\ExtensionTok}[1]{#1}
\newcommand{\FloatTok}[1]{\textcolor[rgb]{0.00,0.00,0.81}{#1}}
\newcommand{\FunctionTok}[1]{\textcolor[rgb]{0.13,0.29,0.53}{\textbf{#1}}}
\newcommand{\ImportTok}[1]{#1}
\newcommand{\InformationTok}[1]{\textcolor[rgb]{0.56,0.35,0.01}{\textbf{\textit{#1}}}}
\newcommand{\KeywordTok}[1]{\textcolor[rgb]{0.13,0.29,0.53}{\textbf{#1}}}
\newcommand{\NormalTok}[1]{#1}
\newcommand{\OperatorTok}[1]{\textcolor[rgb]{0.81,0.36,0.00}{\textbf{#1}}}
\newcommand{\OtherTok}[1]{\textcolor[rgb]{0.56,0.35,0.01}{#1}}
\newcommand{\PreprocessorTok}[1]{\textcolor[rgb]{0.56,0.35,0.01}{\textit{#1}}}
\newcommand{\RegionMarkerTok}[1]{#1}
\newcommand{\SpecialCharTok}[1]{\textcolor[rgb]{0.81,0.36,0.00}{\textbf{#1}}}
\newcommand{\SpecialStringTok}[1]{\textcolor[rgb]{0.31,0.60,0.02}{#1}}
\newcommand{\StringTok}[1]{\textcolor[rgb]{0.31,0.60,0.02}{#1}}
\newcommand{\VariableTok}[1]{\textcolor[rgb]{0.00,0.00,0.00}{#1}}
\newcommand{\VerbatimStringTok}[1]{\textcolor[rgb]{0.31,0.60,0.02}{#1}}
\newcommand{\WarningTok}[1]{\textcolor[rgb]{0.56,0.35,0.01}{\textbf{\textit{#1}}}}
\usepackage{graphicx}
\makeatletter
\def\maxwidth{\ifdim\Gin@nat@width>\linewidth\linewidth\else\Gin@nat@width\fi}
\def\maxheight{\ifdim\Gin@nat@height>\textheight\textheight\else\Gin@nat@height\fi}
\makeatother
% Scale images if necessary, so that they will not overflow the page
% margins by default, and it is still possible to overwrite the defaults
% using explicit options in \includegraphics[width, height, ...]{}
\setkeys{Gin}{width=\maxwidth,height=\maxheight,keepaspectratio}
% Set default figure placement to htbp
\makeatletter
\def\fps@figure{htbp}
\makeatother
\setlength{\emergencystretch}{3em} % prevent overfull lines
\providecommand{\tightlist}{%
  \setlength{\itemsep}{0pt}\setlength{\parskip}{0pt}}
\setcounter{secnumdepth}{-\maxdimen} % remove section numbering
\ifLuaTeX
  \usepackage{selnolig}  % disable illegal ligatures
\fi
\usepackage{bookmark}
\IfFileExists{xurl.sty}{\usepackage{xurl}}{} % add URL line breaks if available
\urlstyle{same}
\hypersetup{
  pdftitle={Multitrait BLUP with metan},
  pdfauthor={Maarit Mäenpää},
  hidelinks,
  pdfcreator={LaTeX via pandoc}}

\title{Multitrait BLUP with metan}
\author{Maarit Mäenpää}
\date{2025-02-03}

\begin{document}
\maketitle

A case of a single experiment: One environment, multiple genotypes.

\subsubsection{Packages needed in the
example}\label{packages-needed-in-the-example}

\begin{Shaded}
\begin{Highlighting}[]
\FunctionTok{library}\NormalTok{(agridat) }\CommentTok{\# for dataset}
\FunctionTok{library}\NormalTok{(dplyr)}
\FunctionTok{library}\NormalTok{(metan)}
\FunctionTok{library}\NormalTok{(ggplot2)}
\end{Highlighting}
\end{Shaded}

\section{Prepare an example dataset}\label{prepare-an-example-dataset}

Using barrero.maize from agridat. This is a multi-year, multi-trait,
multi-environment dataset, where for the sake of an example, we extract
one year and one location. We omit all missing values.

\begin{Shaded}
\begin{Highlighting}[]
\CommentTok{\# Load in the data}
\FunctionTok{data}\NormalTok{(barrero.maize)}
\FunctionTok{head}\NormalTok{(barrero.maize)}
\end{Highlighting}
\end{Shaded}

\begin{verbatim}
##   year  yor loc    env rep  gen daystoflower plantheight earheight population
## 1 2000 2000  BA 2000BA  R1 9211           78      233.68     83.82   57407.43
## 2 2000 2000  BA 2000BA  R2 9211           77      236.22     76.20   61712.00
## 3 2000 2000  BA 2000BA  R3 9211           77      228.60     78.74   57407.43
## 4 2000 2000  BA 2000BA  R4 9211           78      241.30     88.90   60278.79
## 5 2000 2000  BA 2000BA  R1 9114           77      243.84     88.90   59559.72
## 6 2000 2000  BA 2000BA  R2 9114           78      236.22     78.74   58124.04
##   lodged moisture testweight    yield
## 1      0     11.2   81.40275 9.791345
## 2      0     10.7   81.41562 9.481665
## 3      0     10.7   80.38602 8.690092
## 4      0     11.3   83.10159 9.247033
## 5      0     10.1         NA 7.968912
## 6      0     10.3   79.00893 9.361655
\end{verbatim}

\begin{Shaded}
\begin{Highlighting}[]
\CommentTok{\# For details of the data}
\CommentTok{\#?barrero.maize}

\CommentTok{\# Select one location in one year}
\NormalTok{maize.BA }\OtherTok{\textless{}{-}} \FunctionTok{filter}\NormalTok{(barrero.maize, loc}\SpecialCharTok{==}\StringTok{"BA"} \SpecialCharTok{\&}\NormalTok{ year}\SpecialCharTok{==}\StringTok{"2000"}\NormalTok{)}

\CommentTok{\# Omit all NA\textquotesingle{}s}
\NormalTok{maize.BA }\OtherTok{\textless{}{-}} \FunctionTok{na.omit}\NormalTok{(maize.BA)}

\CommentTok{\# Make sure genotypes that are removed due to missing data are not stored as a level in the genotype by reassigning the factor}
\NormalTok{maize.BA}\SpecialCharTok{$}\NormalTok{gen }\OtherTok{\textless{}{-}} \FunctionTok{factor}\NormalTok{(maize.BA}\SpecialCharTok{$}\NormalTok{gen)}
\end{Highlighting}
\end{Shaded}

\section{Run a model with metan}\label{run-a-model-with-metan}

Need to specify the dataset (maize.BA), the column identifying the
genotype (gen), replication (rep), and the response variables of
interest. Here, there are 8 traits measured.

\begin{Shaded}
\begin{Highlighting}[]
\NormalTok{mod }\OtherTok{\textless{}{-}}
  \FunctionTok{gamem}\NormalTok{(maize.BA, }
        \AttributeTok{gen=}\NormalTok{gen, }
        \AttributeTok{rep=}\NormalTok{rep, }
        \AttributeTok{resp=}\FunctionTok{c}\NormalTok{(daystoflower, plantheight, earheight, population, lodged, moisture, testweight, yield))}
\end{Highlighting}
\end{Shaded}

\begin{verbatim}
## Evaluating trait daystoflower |====                              | 12% 00:00:00 Evaluating trait plantheight |=========                          | 25% 00:00:00 Evaluating trait earheight |==============                       | 38% 00:00:00 Evaluating trait population |==================                  | 50% 00:00:00 Evaluating trait lodged |=========================               | 62% 00:00:00 Evaluating trait moisture |============================          | 75% 00:00:00 Evaluating trait testweight |================================    | 88% 00:00:00 Evaluating trait yield |=========================================| 100% 00:00:00 
\end{verbatim}

\begin{verbatim}
## ---------------------------------------------------------------------------
## P-values for Likelihood Ratio Test of the analyzed traits
## ---------------------------------------------------------------------------
##     model daystoflower plantheight earheight population lodged moisture
##  Complete           NA          NA        NA         NA     NA       NA
##  Genotype     2.46e-27    4.24e-29  1.67e-15    0.00945 0.0638 3.03e-14
##  testweight    yield
##          NA       NA
##    0.000104 4.72e-10
## ---------------------------------------------------------------------------
## Variables with nonsignificant Genotype effect
## lodged 
## ---------------------------------------------------------------------------
\end{verbatim}

Note that the output after running the model indicates a brief overview
of which traits differ between genotypes. In this case only the trait
``lodged'' did not differentiate between different genotypes.

\section{BLUPs}\label{blups}

Plot the individual BLUPs (best linear unbiased predictors) from the
model.

\begin{Shaded}
\begin{Highlighting}[]
\FunctionTok{plot\_blup}\NormalTok{(mod, }\AttributeTok{var =} \StringTok{"daystoflower"}\NormalTok{)}
\end{Highlighting}
\end{Shaded}

\includegraphics{Multitrait_BLUPs_with_metan_files/figure-latex/unnamed-chunk-4-1.pdf}

\begin{Shaded}
\begin{Highlighting}[]
\FunctionTok{plot\_blup}\NormalTok{(mod, }\AttributeTok{var =} \StringTok{"plantheight"}\NormalTok{)}
\end{Highlighting}
\end{Shaded}

\includegraphics{Multitrait_BLUPs_with_metan_files/figure-latex/unnamed-chunk-4-2.pdf}

\begin{Shaded}
\begin{Highlighting}[]
\FunctionTok{plot\_blup}\NormalTok{(mod, }\AttributeTok{var =} \StringTok{"earheight"}\NormalTok{)}
\end{Highlighting}
\end{Shaded}

\includegraphics{Multitrait_BLUPs_with_metan_files/figure-latex/unnamed-chunk-4-3.pdf}

\begin{Shaded}
\begin{Highlighting}[]
\FunctionTok{plot\_blup}\NormalTok{(mod, }\AttributeTok{var =} \StringTok{"population"}\NormalTok{)}
\end{Highlighting}
\end{Shaded}

\includegraphics{Multitrait_BLUPs_with_metan_files/figure-latex/unnamed-chunk-4-4.pdf}

\begin{Shaded}
\begin{Highlighting}[]
\FunctionTok{plot\_blup}\NormalTok{(mod, }\AttributeTok{var =} \StringTok{"lodged"}\NormalTok{)}
\end{Highlighting}
\end{Shaded}

\includegraphics{Multitrait_BLUPs_with_metan_files/figure-latex/unnamed-chunk-4-5.pdf}

\begin{Shaded}
\begin{Highlighting}[]
\FunctionTok{plot\_blup}\NormalTok{(mod, }\AttributeTok{var =} \StringTok{"moisture"}\NormalTok{)}
\end{Highlighting}
\end{Shaded}

\includegraphics{Multitrait_BLUPs_with_metan_files/figure-latex/unnamed-chunk-4-6.pdf}

\begin{Shaded}
\begin{Highlighting}[]
\FunctionTok{plot\_blup}\NormalTok{(mod, }\AttributeTok{var =} \StringTok{"testweight"}\NormalTok{)}
\end{Highlighting}
\end{Shaded}

\includegraphics{Multitrait_BLUPs_with_metan_files/figure-latex/unnamed-chunk-4-7.pdf}

\begin{Shaded}
\begin{Highlighting}[]
\FunctionTok{plot\_blup}\NormalTok{(mod, }\AttributeTok{var =} \StringTok{"yield"}\NormalTok{)}
\end{Highlighting}
\end{Shaded}

\includegraphics{Multitrait_BLUPs_with_metan_files/figure-latex/unnamed-chunk-4-8.pdf}

\section{Assess the model}\label{assess-the-model}

\subsubsection{Broad sense heritability}\label{broad-sense-heritability}

Proportion of phenotypic variance due to genetic factors.

\begin{Shaded}
\begin{Highlighting}[]
\FunctionTok{get\_model\_data}\NormalTok{(mod, }\StringTok{"h2"}\NormalTok{) }\CommentTok{\# broad{-}sense heritability}
\end{Highlighting}
\end{Shaded}

\begin{verbatim}
## Class of the model: gamem
\end{verbatim}

\begin{verbatim}
## Variable extracted: h2
\end{verbatim}

\begin{verbatim}
##            VAR        h2
## 1 daystoflower 0.9439548
## 2  plantheight 0.9466682
## 3    earheight 0.8597853
## 4   population 0.5155175
## 5       lodged 0.3992717
## 6     moisture 0.8723022
## 7   testweight 0.6563036
## 8        yield 0.8015240
\end{verbatim}

\subsubsection{Model details}\label{model-details}

\textbf{Ngen} - The number of genetic entries or genotypes evaluated.

\textbf{OVmean} - The overall mean value for each trait across all
genotypes.

\textbf{Min} - The minimum observed value for each trait, with the
genotype and replication noted.

\textbf{Max} - The maximum observed value for each trait, with the
genotype and replication noted.

\textbf{MinGEN} - The minimum genetic value for each trait, indicating
the lowest value among the genotypes.

\textbf{MaxGEN} - The maximum genetic value for each trait, indicating
the highest value among the genotypes.

\begin{Shaded}
\begin{Highlighting}[]
\FunctionTok{get\_model\_data}\NormalTok{(mod, }\StringTok{"details"}\NormalTok{) }\CommentTok{\# General details}
\end{Highlighting}
\end{Shaded}

\begin{verbatim}
## Class of the model: gamem
\end{verbatim}

\begin{verbatim}
## Variable extracted: details
\end{verbatim}

\begin{verbatim}
## # A tibble: 6 x 9
##   Parameters daystoflower      plantheight  earheight population lodged moisture
##   <chr>      <chr>             <chr>        <chr>     <chr>      <chr>  <chr>   
## 1 Ngen       34                34           34        34         34     34      
## 2 OVmean     79.2923           242.1988     90.5998   60465.5671 0.1615 11.1893 
## 3 Min        76 (32P76 in R1)  210.82 (834~ 63.5 (RX~ 48079.219~ 0 (92~ 10.1 (9~
## 4 Max        83 (2K208Y in R2) 269.24 (31B~ 114.3 (9~ 68887.931~ 3 (TR~ 13.7 (2~
## 5 MinGEN     77.25 (DK679BtY)  214.63 (RX8~ 67.945 (~ 55254.531~ 0 (28~ 10.2333~
## 6 MaxGEN     82.5 (2K208Y)     262.4667 (3~ 108.585 ~ 65480.971~ 1.25 ~ 13.4 (2~
## # i 2 more variables: testweight <chr>, yield <chr>
\end{verbatim}

\subsubsection{Likelihood ratio tests}\label{likelihood-ratio-tests}

These tests give the p-values (remember, p-values are not everything!)
for whether the traits assessed show differences in the genotypes.

\begin{Shaded}
\begin{Highlighting}[]
\FunctionTok{get\_model\_data}\NormalTok{(mod, }\StringTok{"lrt"}\NormalTok{) }\CommentTok{\# P{-}values for genotype differences within traits}
\end{Highlighting}
\end{Shaded}

\begin{verbatim}
## Class of the model: gamem
\end{verbatim}

\begin{verbatim}
## Variable extracted: lrt
\end{verbatim}

\begin{verbatim}
## # A tibble: 8 x 8
##   VAR          model     npar logLik   AIC    LRT    Df `Pr(>Chisq)`
##   <chr>        <chr>    <dbl>  <dbl> <dbl>  <dbl> <dbl>        <dbl>
## 1 daystoflower Genotype     5  -244.  499. 117.       1     2.46e-27
## 2 plantheight  Genotype     5  -501. 1013. 125.       1     4.24e-29
## 3 earheight    Genotype     5  -488.  986.  63.4      1     1.67e-15
## 4 population   Genotype     5 -1215. 2440.   6.74     1     9.45e- 3
## 5 lodged       Genotype     5  -107.  223.   3.44     1     6.38e- 2
## 6 moisture     Genotype     5  -153.  316.  57.7      1     3.03e-14
## 7 testweight   Genotype     5  -274.  559.  15.1      1     1.04e- 4
## 8 yield        Genotype     5  -176.  362.  38.8      1     4.72e-10
\end{verbatim}

\subsubsection{Genetic parameters}\label{genetic-parameters}

\textbf{Gen\_var} - Genetic variance, indicating the variability in
traits due to genetic differences.

\textbf{Gen (\%)} - Percentage of total variance attributed to genetic
factors.

\textbf{Res\_var} - Residual variance, representing the variability not
explained by genetic or block effects.

\textbf{Res (\%)} - Percentage of total variance attributed to residual
factors.

\textbf{Phen\_var} - Phenotypic variance, the total observed variability
in the traits.

\textbf{H2} - Broad-sense heritability, indicating the proportion of
phenotypic variance due to genetic factors.

\textbf{h2mg} - Narrow-sense heritability, showing the proportion of
phenotypic variance due to additive genetic factors.

\textbf{Accuracy} - The accuracy of the heritability estimates.

\textbf{CVg} - Coefficient of genetic variation, a measure of genetic
diversity relative to the mean.

\textbf{CVr} - Coefficient of residual variation, indicating the
residual variability relative to the mean.

\textbf{CV ratio} - Ratio of genetic to residual variation, providing
insight into the relative contributions of genetic and residual factors

\begin{Shaded}
\begin{Highlighting}[]
\FunctionTok{get\_model\_data}\NormalTok{(mod, }\StringTok{"genpar"}\NormalTok{) }\CommentTok{\# Genetic parameters}
\end{Highlighting}
\end{Shaded}

\begin{verbatim}
## Class of the model: gamem
\end{verbatim}

\begin{verbatim}
## Variable extracted: genpar
\end{verbatim}

\begin{verbatim}
## # A tibble: 11 x 9
##    Parameters daystoflower plantheight earheight   population   lodged moisture
##    <chr>             <dbl>       <dbl>     <dbl>        <dbl>    <dbl>    <dbl>
##  1 Gen_var           2.02      116.       68.2    2614067.      0.0406    0.381
##  2 Gen (%)          80.8        81.6      60.5         21.0    14.2      63.1  
##  3 Res_var           0.480      26.2      44.5    9826783.      0.244     0.223
##  4 Res (%)          19.2        18.4      39.5         79.0    85.8      36.9  
##  5 Phen_var          2.50      143.      113.    12440850.      0.285     0.605
##  6 H2                0.808       0.816     0.605        0.210   0.142     0.631
##  7 h2mg              0.944       0.947     0.860        0.516   0.399     0.872
##  8 Accuracy          0.972       0.973     0.927        0.718   0.632     0.934
##  9 CVg               1.79        4.45      9.11         2.67  125.        5.52 
## 10 CVr               0.873       2.11      7.36         5.18  306.        4.22 
## 11 CV ratio          2.05        2.11      1.24         0.516   0.408     1.31 
## # i 2 more variables: testweight <dbl>, yield <dbl>
\end{verbatim}

\subsubsection{BLUPs}\label{blups-1}

I.e. predicted means (=point estimates for genotype mean in each trait)

\begin{Shaded}
\begin{Highlighting}[]
\FunctionTok{get\_model\_data}\NormalTok{(mod, }\StringTok{"blupg"}\NormalTok{) }\CommentTok{\# Predicted means}
\end{Highlighting}
\end{Shaded}

\begin{verbatim}
## Class of the model: gamem
\end{verbatim}

\begin{verbatim}
## Variable extracted: blupg
\end{verbatim}

\begin{verbatim}
## # A tibble: 34 x 9
##    GEN     daystoflower plantheight earheight population lodged moisture
##    <chr>          <dbl>       <dbl>     <dbl>      <dbl>  <dbl>    <dbl>
##  1 1866Bt          78.6        243.      88.0     60006. 0.395      11.1
##  2 2888IMI         77.9        242.      94.9     59863. 0.128      11.3
##  3 2K103Y          80.3        240.      78.2     58434. 0.0955     13.1
##  4 2K207Y          81.9        240.      78.7     60653. 0.195      10.7
##  5 2K208Y          82.4        257.      96.2     61856. 0.0955     12.3
##  6 2K213Y          77.6        235.      91.8     59421. 0.0980     11.6
##  7 31B13           80.3        257.     102.      60958. 0.0672     11.0
##  8 31R88           79.1        262.      95.3     60589. 0.0980     12.1
##  9 3223            80.6        257.     103.      61650. 0.223      10.9
## 10 32P76           77.6        239.      90.1     61668. 0.128      10.9
## # i 24 more rows
## # i 2 more variables: testweight <dbl>, yield <dbl>
\end{verbatim}

\section{FAI-index}\label{fai-index}

Multitrait comparison. The method is documented first here:
\url{https://doi.org/10.1111/gcbb.12443}

\begin{Shaded}
\begin{Highlighting}[]
\NormalTok{FAI }\OtherTok{\textless{}{-}} \FunctionTok{fai\_blup}\NormalTok{(mod)}
\end{Highlighting}
\end{Shaded}

\begin{verbatim}
## 
## -----------------------------------------------------------------------------------
## Principal Component Analysis
## -----------------------------------------------------------------------------------
##     eigen.values cumulative.var
## PC1         2.85          35.59
## PC2         1.64          56.13
## PC3         1.11          70.01
## PC4         1.05          83.11
## PC5         0.67          91.43
## PC6         0.38          96.23
## PC7         0.17          98.36
## PC8         0.13         100.00
## 
## -----------------------------------------------------------------------------------
## Factor Analysis
## -----------------------------------------------------------------------------------
##                FA1   FA2   FA3   FA4 comunalits
## daystoflower  0.08  0.83 -0.38 -0.10       0.85
## plantheight  -0.69  0.52 -0.20  0.17       0.82
## earheight    -0.86  0.29  0.07  0.16       0.85
## population   -0.18  0.78  0.18 -0.15       0.70
## lodged       -0.05  0.18  0.01 -0.91       0.87
## moisture     -0.04  0.07 -0.96  0.01       0.92
## testweight   -0.69 -0.31 -0.26 -0.35       0.77
## yield        -0.91 -0.03  0.07 -0.19       0.86
## 
## -----------------------------------------------------------------------------------
## Comunalit Mean: 0.8310696 
## Selection differential
## -----------------------------------------------------------------------------------
## # A tibble: 8 x 9
##   VAR          Factor        Xo        Xs      SD SDperc    h2       SG SGperc
##   <chr>         <dbl>     <dbl>     <dbl>   <dbl>  <dbl> <dbl>    <dbl>  <dbl>
## 1 plantheight       1   242.      251.      9.07    3.75 0.947   8.59    3.55 
## 2 earheight         1    90.6      94.2     3.55    3.92 0.860   3.06    3.37 
## 3 testweight        1    79.5      80.4     0.950   1.20 0.656   0.624   0.785
## 4 yield             1     9.56     10.1     0.578   6.04 0.802   0.463   4.84 
## 5 daystoflower      2    79.3      80.6     1.34    1.69 0.944   1.26    1.59 
## 6 population        2 60466.    61195.    730.      1.21 0.516 376.      0.622
## 7 moisture          3    11.2      11.8     0.637   5.69 0.872   0.556   4.97 
## 8 lodged            4     0.162     0.276   0.114  70.7  0.399   0.0456 28.2  
## 
## -----------------------------------------------------------------------------------
## Selected genotypes
## DK697 2K208Y DK687 31R88 1866Bt
## -----------------------------------------------------------------------------------
\end{verbatim}

\begin{Shaded}
\begin{Highlighting}[]
\FunctionTok{plot}\NormalTok{(FAI)}
\end{Highlighting}
\end{Shaded}

\includegraphics{Multitrait_BLUPs_with_metan_files/figure-latex/unnamed-chunk-10-1.pdf}

\subsection{FAI ranking and illustrating selected
genotypes}\label{fai-ranking-and-illustrating-selected-genotypes}

Here, genotype selection is not purely for the optimal top performing
genotypes, but for diversifying the best- and worst performing
genotypes. In this case, we take the top and bottom 2 genotypes, and
randomly choose 4 from the middle.

\begin{Shaded}
\begin{Highlighting}[]
\CommentTok{\# Save BLUP data for individual traits}
\NormalTok{metan\_blups }\OtherTok{\textless{}{-}} \FunctionTok{data.frame}\NormalTok{(}\FunctionTok{get\_model\_data}\NormalTok{(mod, }\StringTok{"ranef"}\NormalTok{)}\SpecialCharTok{$}\NormalTok{GEN)}
\end{Highlighting}
\end{Shaded}

\begin{verbatim}
## Class of the model: gamem
\end{verbatim}

\begin{verbatim}
## Variable extracted: ranef
\end{verbatim}

\begin{Shaded}
\begin{Highlighting}[]
\CommentTok{\# Extract FAI indexes}
\NormalTok{FAI\_results }\OtherTok{\textless{}{-}}\NormalTok{ FAI}\SpecialCharTok{$}\NormalTok{FAI}

\CommentTok{\# Change the name of Genotype to combine the two dataframes}
\FunctionTok{names}\NormalTok{(metan\_blups)[}\DecValTok{1}\NormalTok{] }\OtherTok{\textless{}{-}} \StringTok{"Genotype"}

\CommentTok{\# Combine dataframes}
\NormalTok{FAI\_results }\OtherTok{\textless{}{-}} \FunctionTok{left\_join}\NormalTok{(FAI\_results, metan\_blups, }\AttributeTok{by=}\StringTok{"Genotype"}\NormalTok{)}

\CommentTok{\# Create a dataframe to select genotypes}
\NormalTok{selected }\OtherTok{\textless{}{-}} \FunctionTok{data.frame}\NormalTok{(}\FunctionTok{matrix}\NormalTok{(}\ConstantTok{NA}\NormalTok{, }\AttributeTok{nrow =} \DecValTok{0}\NormalTok{, }\AttributeTok{ncol =} \DecValTok{25}\NormalTok{))}
\FunctionTok{colnames}\NormalTok{(selected) }\OtherTok{\textless{}{-}} \FunctionTok{colnames}\NormalTok{(FAI\_results)}

\CommentTok{\# Top and bottom 5}
\NormalTok{selected }\OtherTok{\textless{}{-}} \FunctionTok{rbind}\NormalTok{(selected, }
\NormalTok{                  FAI\_results[}\FunctionTok{which}\NormalTok{(FAI\_results}\SpecialCharTok{$}\NormalTok{ID1 }\SpecialCharTok{\%in\%} \FunctionTok{tail}\NormalTok{(}\FunctionTok{sort}\NormalTok{(FAI\_results}\SpecialCharTok{$}\NormalTok{ID1),}\DecValTok{2}\NormalTok{)}\SpecialCharTok{==}\NormalTok{T),]}
\NormalTok{)}
\NormalTok{selected }\OtherTok{\textless{}{-}} \FunctionTok{rbind}\NormalTok{(selected, }
\NormalTok{                  FAI\_results[}\FunctionTok{which}\NormalTok{(FAI\_results}\SpecialCharTok{$}\NormalTok{ID1 }\SpecialCharTok{\%in\%} \FunctionTok{head}\NormalTok{(}\FunctionTok{sort}\NormalTok{(FAI\_results}\SpecialCharTok{$}\NormalTok{ID1),}\DecValTok{2}\NormalTok{)}\SpecialCharTok{==}\NormalTok{T),]}
\NormalTok{)}

\CommentTok{\# Random selection from the middle}
\NormalTok{rselect }\OtherTok{\textless{}{-}} \FunctionTok{sample}\NormalTok{(FAI\_results[}\FunctionTok{which}\NormalTok{(FAI\_results}\SpecialCharTok{$}\NormalTok{Genotype }\SpecialCharTok{\%in\%}\NormalTok{ selected}\SpecialCharTok{$}\NormalTok{Genotype}\SpecialCharTok{==}\ConstantTok{FALSE}\NormalTok{),]}\SpecialCharTok{$}\NormalTok{Genotype, }\AttributeTok{size =} \DecValTok{4}\NormalTok{, }\AttributeTok{replace =}\NormalTok{ F)}
\NormalTok{selected }\OtherTok{\textless{}{-}} \FunctionTok{rbind}\NormalTok{(selected, FAI\_results[}\FunctionTok{which}\NormalTok{(FAI\_results}\SpecialCharTok{$}\NormalTok{Genotype }\SpecialCharTok{\%in\%}\NormalTok{ rselect }\SpecialCharTok{==}\NormalTok{T),])}

\CommentTok{\# Illustrate selection}
\FunctionTok{ggplot}\NormalTok{(FAI\_results, }\FunctionTok{aes}\NormalTok{(}\AttributeTok{x=}\NormalTok{ID2, }\AttributeTok{y=}\NormalTok{ID1)) }\SpecialCharTok{+} 
  \FunctionTok{geom\_point}\NormalTok{() }\SpecialCharTok{+} 
  \FunctionTok{geom\_point}\NormalTok{(}\AttributeTok{data=}\NormalTok{selected, }\AttributeTok{colour=}\StringTok{"red"}\NormalTok{)}
\end{Highlighting}
\end{Shaded}

\includegraphics{Multitrait_BLUPs_with_metan_files/figure-latex/unnamed-chunk-11-1.pdf}

The figure above illustrates the ranked order of the genotypes. The red
are selected, and black are not. Distance in ID1 and ID2 illustrate
difference between the genotypes.

\end{document}
